% Options for packages loaded elsewhere
\PassOptionsToPackage{unicode}{hyperref}
\PassOptionsToPackage{hyphens}{url}
%
\documentclass[
]{article}
\usepackage{lmodern}
\usepackage{amssymb,amsmath}
\usepackage{ifxetex,ifluatex}
\ifnum 0\ifxetex 1\fi\ifluatex 1\fi=0 % if pdftex
  \usepackage[T1]{fontenc}
  \usepackage[utf8]{inputenc}
  \usepackage{textcomp} % provide euro and other symbols
\else % if luatex or xetex
  \usepackage{unicode-math}
  \defaultfontfeatures{Scale=MatchLowercase}
  \defaultfontfeatures[\rmfamily]{Ligatures=TeX,Scale=1}
\fi
% Use upquote if available, for straight quotes in verbatim environments
\IfFileExists{upquote.sty}{\usepackage{upquote}}{}
\IfFileExists{microtype.sty}{% use microtype if available
  \usepackage[]{microtype}
  \UseMicrotypeSet[protrusion]{basicmath} % disable protrusion for tt fonts
}{}
\makeatletter
\@ifundefined{KOMAClassName}{% if non-KOMA class
  \IfFileExists{parskip.sty}{%
    \usepackage{parskip}
  }{% else
    \setlength{\parindent}{0pt}
    \setlength{\parskip}{6pt plus 2pt minus 1pt}}
}{% if KOMA class
  \KOMAoptions{parskip=half}}
\makeatother
\usepackage{xcolor}
\IfFileExists{xurl.sty}{\usepackage{xurl}}{} % add URL line breaks if available
\IfFileExists{bookmark.sty}{\usepackage{bookmark}}{\usepackage{hyperref}}
\hypersetup{
  pdftitle={STAT 147 \textbar{} HW},
  pdfauthor={Victoria Nguyen},
  hidelinks,
  pdfcreator={LaTeX via pandoc}}
\urlstyle{same} % disable monospaced font for URLs
\usepackage[margin=1in]{geometry}
\usepackage{color}
\usepackage{fancyvrb}
\newcommand{\VerbBar}{|}
\newcommand{\VERB}{\Verb[commandchars=\\\{\}]}
\DefineVerbatimEnvironment{Highlighting}{Verbatim}{commandchars=\\\{\}}
% Add ',fontsize=\small' for more characters per line
\usepackage{framed}
\definecolor{shadecolor}{RGB}{248,248,248}
\newenvironment{Shaded}{\begin{snugshade}}{\end{snugshade}}
\newcommand{\AlertTok}[1]{\textcolor[rgb]{0.94,0.16,0.16}{#1}}
\newcommand{\AnnotationTok}[1]{\textcolor[rgb]{0.56,0.35,0.01}{\textbf{\textit{#1}}}}
\newcommand{\AttributeTok}[1]{\textcolor[rgb]{0.77,0.63,0.00}{#1}}
\newcommand{\BaseNTok}[1]{\textcolor[rgb]{0.00,0.00,0.81}{#1}}
\newcommand{\BuiltInTok}[1]{#1}
\newcommand{\CharTok}[1]{\textcolor[rgb]{0.31,0.60,0.02}{#1}}
\newcommand{\CommentTok}[1]{\textcolor[rgb]{0.56,0.35,0.01}{\textit{#1}}}
\newcommand{\CommentVarTok}[1]{\textcolor[rgb]{0.56,0.35,0.01}{\textbf{\textit{#1}}}}
\newcommand{\ConstantTok}[1]{\textcolor[rgb]{0.00,0.00,0.00}{#1}}
\newcommand{\ControlFlowTok}[1]{\textcolor[rgb]{0.13,0.29,0.53}{\textbf{#1}}}
\newcommand{\DataTypeTok}[1]{\textcolor[rgb]{0.13,0.29,0.53}{#1}}
\newcommand{\DecValTok}[1]{\textcolor[rgb]{0.00,0.00,0.81}{#1}}
\newcommand{\DocumentationTok}[1]{\textcolor[rgb]{0.56,0.35,0.01}{\textbf{\textit{#1}}}}
\newcommand{\ErrorTok}[1]{\textcolor[rgb]{0.64,0.00,0.00}{\textbf{#1}}}
\newcommand{\ExtensionTok}[1]{#1}
\newcommand{\FloatTok}[1]{\textcolor[rgb]{0.00,0.00,0.81}{#1}}
\newcommand{\FunctionTok}[1]{\textcolor[rgb]{0.00,0.00,0.00}{#1}}
\newcommand{\ImportTok}[1]{#1}
\newcommand{\InformationTok}[1]{\textcolor[rgb]{0.56,0.35,0.01}{\textbf{\textit{#1}}}}
\newcommand{\KeywordTok}[1]{\textcolor[rgb]{0.13,0.29,0.53}{\textbf{#1}}}
\newcommand{\NormalTok}[1]{#1}
\newcommand{\OperatorTok}[1]{\textcolor[rgb]{0.81,0.36,0.00}{\textbf{#1}}}
\newcommand{\OtherTok}[1]{\textcolor[rgb]{0.56,0.35,0.01}{#1}}
\newcommand{\PreprocessorTok}[1]{\textcolor[rgb]{0.56,0.35,0.01}{\textit{#1}}}
\newcommand{\RegionMarkerTok}[1]{#1}
\newcommand{\SpecialCharTok}[1]{\textcolor[rgb]{0.00,0.00,0.00}{#1}}
\newcommand{\SpecialStringTok}[1]{\textcolor[rgb]{0.31,0.60,0.02}{#1}}
\newcommand{\StringTok}[1]{\textcolor[rgb]{0.31,0.60,0.02}{#1}}
\newcommand{\VariableTok}[1]{\textcolor[rgb]{0.00,0.00,0.00}{#1}}
\newcommand{\VerbatimStringTok}[1]{\textcolor[rgb]{0.31,0.60,0.02}{#1}}
\newcommand{\WarningTok}[1]{\textcolor[rgb]{0.56,0.35,0.01}{\textbf{\textit{#1}}}}
\usepackage{graphicx,grffile}
\makeatletter
\def\maxwidth{\ifdim\Gin@nat@width>\linewidth\linewidth\else\Gin@nat@width\fi}
\def\maxheight{\ifdim\Gin@nat@height>\textheight\textheight\else\Gin@nat@height\fi}
\makeatother
% Scale images if necessary, so that they will not overflow the page
% margins by default, and it is still possible to overwrite the defaults
% using explicit options in \includegraphics[width, height, ...]{}
\setkeys{Gin}{width=\maxwidth,height=\maxheight,keepaspectratio}
% Set default figure placement to htbp
\makeatletter
\def\fps@figure{htbp}
\makeatother
\setlength{\emergencystretch}{3em} % prevent overfull lines
\providecommand{\tightlist}{%
  \setlength{\itemsep}{0pt}\setlength{\parskip}{0pt}}
\setcounter{secnumdepth}{-\maxdimen} % remove section numbering

\title{STAT 147 \textbar{} HW}
\usepackage{etoolbox}
\makeatletter
\providecommand{\subtitle}[1]{% add subtitle to \maketitle
  \apptocmd{\@title}{\par {\large #1 \par}}{}{}
}
\makeatother
\subtitle{Week 3 HW}
\author{Victoria Nguyen}
\date{}

\begin{document}
\maketitle

\begin{center}\rule{0.5\linewidth}{0.5pt}\end{center}

\hypertarget{instructions}{%
\section{\texorpdfstring{\textbf{Instructions}}{Instructions}}\label{instructions}}

The goal of this assignment is to continue our exploration of loops,
while loops, and if and else statements. Please use the RMD file
\texttt{FIRST\_LAST\_HW\_03.Rmd} to answer your questions. Delete code
or answer chunks as necessary. You will submit both the RMD file and the
corresponding HTML file from the knitted document.

Note: you may need to search the internet to find answers for some of
the questions.

You will be graded as follows:

\begin{itemize}
\tightlist
\item
  Does your R chunks run (some errors are acceptable in this
  assignment)?
\item
  Have you completed the assignment in its entirety?
\item
  Have you followed the instructions carefully?
\item
  Have you responded to the questions correctly?
\end{itemize}

\begin{center}\rule{0.5\linewidth}{0.5pt}\end{center}

\hypertarget{grading}{%
\paragraph{Grading}\label{grading}}

\begin{itemize}
\item
  \textbf{(20 pts)} Have you completed the assignment in its entirety?
\item
  \textbf{(20 pts)} Are your responses correct for the subset of
  randomly graded questions?
\item
  \textbf{(5 pts)} CODE DOCUMENTATION. Add comments to nearly every line
  explaining what the line of code is doing.
\item
  \textbf{(5 pts)} Submitted Knitted Document (HTML File)
\end{itemize}

\begin{center}\rule{0.5\linewidth}{0.5pt}\end{center}

\hypertarget{questions}{%
\section{Questions}\label{questions}}

\hypertarget{part-1}{%
\subsection{Part 1}\label{part-1}}

\hypertarget{q1}{%
\subsubsection{Q1}\label{q1}}

\hypertarget{part-a}{%
\paragraph{Part A}\label{part-a}}

Is to be completed in Lab 3B. Below is a description of what is
necessary from the lab.

\begin{itemize}
\item
  Generating a vector filled with Fibonacci Numbers.
\item
  Generating a vector filled with Triangular Numbers.
\item
  Generating a vector filled with Tetrahedral Numbers.
\end{itemize}

\begin{Shaded}
\begin{Highlighting}[]
\CommentTok{#Fibonacci}
\NormalTok{fibonacci <-}\StringTok{ }\KeywordTok{c}\NormalTok{(}\DecValTok{0}\NormalTok{,}\DecValTok{1}\NormalTok{)               }\CommentTok{#initialize the first two numbers: 0 & 1 in a }
\ControlFlowTok{for}\NormalTok{(i }\ControlFlowTok{in}\NormalTok{ (}\DecValTok{3}\OperatorTok{:}\DecValTok{50}\NormalTok{))\{                 }\CommentTok{#for loop for the 3rd to 50th numbers in the sequence}
\NormalTok{  fibonacci[i] <-}\StringTok{ }\NormalTok{(fibonacci[i}\DecValTok{-1}\NormalTok{]}\OperatorTok{+}\NormalTok{fibonacci[i}\DecValTok{-2}\NormalTok{]) }\CommentTok{#each number is found using the two before it}
\NormalTok{\}}
\NormalTok{fibonacci }
\end{Highlighting}
\end{Shaded}

\begin{verbatim}
##  [1]          0          1          1          2          3          5
##  [7]          8         13         21         34         55         89
## [13]        144        233        377        610        987       1597
## [19]       2584       4181       6765      10946      17711      28657
## [25]      46368      75025     121393     196418     317811     514229
## [31]     832040    1346269    2178309    3524578    5702887    9227465
## [37]   14930352   24157817   39088169   63245986  102334155  165580141
## [43]  267914296  433494437  701408733 1134903170 1836311903 2971215073
## [49] 4807526976 7778742049
\end{verbatim}

\begin{Shaded}
\begin{Highlighting}[]
\CommentTok{#Triangular}

\NormalTok{Triangular <-}\StringTok{ }\KeywordTok{c}\NormalTok{(}\DecValTok{1}\NormalTok{) }
\ControlFlowTok{for}\NormalTok{ (i }\ControlFlowTok{in}\NormalTok{ (}\DecValTok{2}\OperatorTok{:}\DecValTok{40}\NormalTok{))}
\NormalTok{  Triangular[i] <-}\StringTok{ }\NormalTok{(i }\OperatorTok{*}\StringTok{ }\NormalTok{(i }\OperatorTok{+}\StringTok{ }\DecValTok{1}\NormalTok{ )) }\OperatorTok{/}\StringTok{ }\DecValTok{2}
  
\CommentTok{#Returns as a vector}
\NormalTok{  Triangular}
\end{Highlighting}
\end{Shaded}

\begin{verbatim}
##  [1]   1   3   6  10  15  21  28  36  45  55  66  78  91 105 120 136 153 171 190
## [20] 210 231 253 276 300 325 351 378 406 435 465 496 528 561 595 630 666 703 741
## [39] 780 820
\end{verbatim}

\begin{Shaded}
\begin{Highlighting}[]
\CommentTok{#Tetrahedral}

\NormalTok{x <-}\StringTok{ }\DecValTok{1}
\NormalTok{z <-}\StringTok{ }\KeywordTok{c}\NormalTok{(}\DecValTok{1}\OperatorTok{:}\DecValTok{39}\NormalTok{)}
\NormalTok{tetrahedral <-}\StringTok{ }\KeywordTok{c}\NormalTok{()}
\ControlFlowTok{for}\NormalTok{ (i }\ControlFlowTok{in} \KeywordTok{seq_along}\NormalTok{(z))\{}
\NormalTok{  tetrahedral[i] <-}\StringTok{ }\NormalTok{(x }\OperatorTok{*}\StringTok{ }\NormalTok{(x }\OperatorTok{+}\StringTok{ }\DecValTok{1}\NormalTok{) }\OperatorTok{*}\StringTok{ }\NormalTok{(x }\OperatorTok{+}\StringTok{ }\DecValTok{2}\NormalTok{)) }\OperatorTok{/}\DecValTok{6}
\NormalTok{  x <-}\StringTok{ }\NormalTok{x }\OperatorTok{+}\StringTok{ }\DecValTok{1}
\NormalTok{\}}
\NormalTok{tetrahedral}
\end{Highlighting}
\end{Shaded}

\begin{verbatim}
##  [1]     1     4    10    20    35    56    84   120   165   220   286   364
## [13]   455   560   680   816   969  1140  1330  1540  1771  2024  2300  2600
## [25]  2925  3276  3654  4060  4495  4960  5456  5984  6545  7140  7770  8436
## [37]  9139  9880 10660
\end{verbatim}

\hypertarget{part-b}{%
\paragraph{Part B}\label{part-b}}

\hypertarget{section}{%
\subparagraph{1.}\label{section}}

Create a vector containing only the even numbers from the Fibonacci
sequence. Hint: Use loops, if else statements and divide each number by
2 using \texttt{\%\%}. Find the sum of all the even Fibonacci numbers?

\begin{Shaded}
\begin{Highlighting}[]
\NormalTok{fibonacci_even <-}\StringTok{ }\KeywordTok{c}\NormalTok{() }\CommentTok{#initialize empty vector}
\NormalTok{n <-}\StringTok{ }\DecValTok{1}\NormalTok{;}
\ControlFlowTok{for}\NormalTok{ (i }\ControlFlowTok{in}\NormalTok{ (}\DecValTok{1}\OperatorTok{:}\DecValTok{50}\NormalTok{) ) \{}
\NormalTok{    x <-}\StringTok{ }\DecValTok{0}
    \CommentTok{#even}
    \ControlFlowTok{if}\NormalTok{ (fibonacci[n]}\OperatorTok\DecValTok{2}\OperatorTok{==}\DecValTok{0}\NormalTok{)}
\NormalTok{    \{}
\NormalTok{        x<-fibonacci[n]}
\NormalTok{    \}}
    \CommentTok{#odd}
    \ControlFlowTok{if}\NormalTok{ (x }\OperatorTok{>}\StringTok{ }\DecValTok{0}\NormalTok{)}
\NormalTok{        fibonacci_even<-}\KeywordTok{c}\NormalTok{(fibonacci_even, x);}
\NormalTok{        n <-}\StringTok{ }\NormalTok{n }\OperatorTok{+}\StringTok{ }\DecValTok{1}\NormalTok{;}
        
\NormalTok{\}}
\NormalTok{fibonacci_even}
\end{Highlighting}
\end{Shaded}

\begin{verbatim}
##  [1]          2          8         34        144        610       2584
##  [7]      10946      46368     196418     832040    3524578   14930352
## [13]   63245986  267914296 1134903170 4807526976
\end{verbatim}

\begin{Shaded}
\begin{Highlighting}[]
\KeywordTok{sum}\NormalTok{(fibonacci_even)}
\end{Highlighting}
\end{Shaded}

\begin{verbatim}
## [1] 6293134512
\end{verbatim}

Sum of the even fibonacci numbers from the first 50: 6293134512

\hypertarget{section-1}{%
\subparagraph{2.}\label{section-1}}

Create a vector containing only the even numbers from the Triangular
sequence. Hint: Use loops, if else statements and divide each number by
2 using \texttt{\%\%}. Find the sum of all the even Triangular numbers?

\begin{Shaded}
\begin{Highlighting}[]
\NormalTok{Tri_even <-}\StringTok{ }\KeywordTok{c}\NormalTok{()}
\NormalTok{n <-}\StringTok{ }\DecValTok{1}\NormalTok{;}
\ControlFlowTok{for}\NormalTok{ (i }\ControlFlowTok{in}\NormalTok{ (}\DecValTok{1}\OperatorTok{:}\DecValTok{40}\NormalTok{) ) \{}
\NormalTok{    x <-}\StringTok{ }\DecValTok{0}
    \CommentTok{#even}
    \ControlFlowTok{if}\NormalTok{ (Triangular[n]}\OperatorTok\DecValTok{2}\OperatorTok{==}\DecValTok{0}\NormalTok{)}
\NormalTok{    \{}
\NormalTok{        x<-Triangular[n]}
\NormalTok{    \}}
    \ControlFlowTok{if}\NormalTok{ (x }\OperatorTok{!=}\StringTok{ }\DecValTok{0}\NormalTok{)}
\NormalTok{        Tri_even<-}\KeywordTok{c}\NormalTok{(Tri_even, x);}
\NormalTok{        n <-}\StringTok{ }\NormalTok{n }\OperatorTok{+}\StringTok{ }\DecValTok{1}\NormalTok{;}
        
\NormalTok{\}}
\NormalTok{Tri_even}
\end{Highlighting}
\end{Shaded}

\begin{verbatim}
##  [1]   6  10  28  36  66  78 120 136 190 210 276 300 378 406 496 528 630 666 780
## [20] 820
\end{verbatim}

\begin{Shaded}
\begin{Highlighting}[]
\KeywordTok{sum}\NormalTok{(Tri_even)}
\end{Highlighting}
\end{Shaded}

\begin{verbatim}
## [1] 6160
\end{verbatim}

Sum of the even triangular numbers: 6160

\hypertarget{section-2}{%
\subparagraph{3.}\label{section-2}}

Create a vector containing only the even numbers from the Tetrahedral
sequence. Hint: Use loops, if else statements and divide each number by
2 using \texttt{\%\%}. Find the sum of all the even Tetrahedral numbers?

\begin{Shaded}
\begin{Highlighting}[]
\NormalTok{tetrahedral_even <-}\StringTok{ }\KeywordTok{c}\NormalTok{()}
\NormalTok{n <-}\StringTok{ }\DecValTok{1}\NormalTok{;}
\ControlFlowTok{for}\NormalTok{ (i }\ControlFlowTok{in}\NormalTok{ (}\DecValTok{1}\OperatorTok{:}\DecValTok{39}\NormalTok{) ) \{}
\NormalTok{    x <-}\StringTok{ }\DecValTok{0}
    \CommentTok{#even}
    \ControlFlowTok{if}\NormalTok{ (tetrahedral[n]}\OperatorTok\DecValTok{2}\OperatorTok{==}\DecValTok{0}\NormalTok{)}
\NormalTok{    \{}
\NormalTok{        x<-tetrahedral[n]}
\NormalTok{    \}}
    \ControlFlowTok{if}\NormalTok{ (x }\OperatorTok{!=}\StringTok{ }\DecValTok{0}\NormalTok{)}
\NormalTok{        tetrahedral_even<-}\KeywordTok{c}\NormalTok{(tetrahedral_even, x);}
\NormalTok{        n <-}\StringTok{ }\NormalTok{n }\OperatorTok{+}\StringTok{ }\DecValTok{1}\NormalTok{;}
        
\NormalTok{\}}
\NormalTok{tetrahedral_even}
\end{Highlighting}
\end{Shaded}

\begin{verbatim}
##  [1]     4    10    20    56    84   120   220   286   364   560   680   816
## [13]  1140  1330  1540  2024  2300  2600  3276  3654  4060  4960  5456  5984
## [25]  7140  7770  8436  9880 10660
\end{verbatim}

\begin{Shaded}
\begin{Highlighting}[]
\KeywordTok{sum}\NormalTok{(tetrahedral_even)}
\end{Highlighting}
\end{Shaded}

\begin{verbatim}
## [1] 85430
\end{verbatim}

sum of even tetrahedral numbers = 85430

\hypertarget{part-2}{%
\subsection{Part 2}\label{part-2}}

Use the following matrix for Part 2 questions:

\begin{Shaded}
\begin{Highlighting}[]
\NormalTok{x <-}\StringTok{ }\KeywordTok{matrix}\NormalTok{(}\DecValTok{1}\OperatorTok{:}\DecValTok{100}\NormalTok{, }\DataTypeTok{nrow =} \DecValTok{10}\NormalTok{)}
\NormalTok{x}
\end{Highlighting}
\end{Shaded}

\begin{verbatim}
##       [,1] [,2] [,3] [,4] [,5] [,6] [,7] [,8] [,9] [,10]
##  [1,]    1   11   21   31   41   51   61   71   81    91
##  [2,]    2   12   22   32   42   52   62   72   82    92
##  [3,]    3   13   23   33   43   53   63   73   83    93
##  [4,]    4   14   24   34   44   54   64   74   84    94
##  [5,]    5   15   25   35   45   55   65   75   85    95
##  [6,]    6   16   26   36   46   56   66   76   86    96
##  [7,]    7   17   27   37   47   57   67   77   87    97
##  [8,]    8   18   28   38   48   58   68   78   88    98
##  [9,]    9   19   29   39   49   59   69   79   89    99
## [10,]   10   20   30   40   50   60   70   80   90   100
\end{verbatim}

\hypertarget{q2}{%
\subsubsection{Q2}\label{q2}}

The \texttt{colSums} function will add all the values in each column of
a matrix. Instead of using the \texttt{colSums} function, write a loop
that will produce the same results.

\begin{Shaded}
\begin{Highlighting}[]
\CommentTok{#initialize 10 x 10 matrix}
\NormalTok{x <-}\StringTok{ }\KeywordTok{matrix}\NormalTok{(}\DecValTok{1}\OperatorTok{:}\DecValTok{100}\NormalTok{, }\DataTypeTok{nrow =} \DecValTok{10}\NormalTok{)}

\CommentTok{#loop for sums across 10 columns}
\NormalTok{colSums_loop <-}\StringTok{ }\KeywordTok{vector}\NormalTok{(}\DataTypeTok{mode =} \StringTok{"numeric"}\NormalTok{)}
\ControlFlowTok{for}\NormalTok{(i }\ControlFlowTok{in} \DecValTok{1}\OperatorTok{:}\DecValTok{10}\NormalTok{) \{}
\NormalTok{  colSums_loop[i] <-}\StringTok{ }\KeywordTok{sum}\NormalTok{(x[}\DecValTok{1}\OperatorTok{:}\DecValTok{10}\NormalTok{, i])}
\NormalTok{\}}
\NormalTok{colSums_loop}
\end{Highlighting}
\end{Shaded}

\begin{verbatim}
##  [1]  55 155 255 355 455 555 655 755 855 955
\end{verbatim}

\hypertarget{q3}{%
\subsubsection{Q3}\label{q3}}

The \texttt{rowMeans} function will compute the mean for each row in a
matrix. Instead of using the \texttt{rowMeans} function, write a loop
that will produce the same results.

\begin{Shaded}
\begin{Highlighting}[]
\CommentTok{#get mean across 10 rows, uses same x as previous question}
\NormalTok{rowMeans_loop <-}\StringTok{ }\KeywordTok{vector}\NormalTok{(}\DataTypeTok{mode =} \StringTok{"numeric"}\NormalTok{)}
\ControlFlowTok{for}\NormalTok{(i }\ControlFlowTok{in} \DecValTok{1}\OperatorTok{:}\DecValTok{10}\NormalTok{) \{}
\NormalTok{  rowMeans_loop[i] <-}\StringTok{ }\KeywordTok{mean}\NormalTok{(x[i, }\DecValTok{1}\OperatorTok{:}\DecValTok{10}\NormalTok{])}
\NormalTok{\}}
\NormalTok{rowMeans_loop}
\end{Highlighting}
\end{Shaded}

\begin{verbatim}
##  [1] 46 47 48 49 50 51 52 53 54 55
\end{verbatim}

\hypertarget{q4}{%
\subsubsection{Q4}\label{q4}}

Create a logical \(10 \times 10\) matrix indicating whether each element
in matrix \texttt{x} is either even (TRUE) or odd (FALSE).

\begin{Shaded}
\begin{Highlighting}[]
\NormalTok{x <-}\StringTok{ }\KeywordTok{matrix}\NormalTok{(}\DecValTok{1}\OperatorTok{:}\DecValTok{100}\NormalTok{, }\DataTypeTok{nrow =} \DecValTok{10}\NormalTok{)}
\NormalTok{x}
\end{Highlighting}
\end{Shaded}

\begin{verbatim}
##       [,1] [,2] [,3] [,4] [,5] [,6] [,7] [,8] [,9] [,10]
##  [1,]    1   11   21   31   41   51   61   71   81    91
##  [2,]    2   12   22   32   42   52   62   72   82    92
##  [3,]    3   13   23   33   43   53   63   73   83    93
##  [4,]    4   14   24   34   44   54   64   74   84    94
##  [5,]    5   15   25   35   45   55   65   75   85    95
##  [6,]    6   16   26   36   46   56   66   76   86    96
##  [7,]    7   17   27   37   47   57   67   77   87    97
##  [8,]    8   18   28   38   48   58   68   78   88    98
##  [9,]    9   19   29   39   49   59   69   79   89    99
## [10,]   10   20   30   40   50   60   70   80   90   100
\end{verbatim}

\begin{Shaded}
\begin{Highlighting}[]
\CommentTok{#empty 10 x 10 matrix}
\NormalTok{x_logical <-}\StringTok{ }\KeywordTok{matrix}\NormalTok{(}\DataTypeTok{nrow =} \DecValTok{10}\NormalTok{, }\DataTypeTok{ncol =} \DecValTok{10}\NormalTok{ ) }
\NormalTok{n <-}\StringTok{ }\DecValTok{1}
\ControlFlowTok{for}\NormalTok{ (i }\ControlFlowTok{in} \KeywordTok{seq_along}\NormalTok{(x) ) \{}
\NormalTok{    num <-}\StringTok{ }\NormalTok{x[i]}
    \CommentTok{#even}
    \ControlFlowTok{if}\NormalTok{ (num}\OperatorTok\DecValTok{2}\OperatorTok{==}\DecValTok{0}\NormalTok{)}
\NormalTok{    \{}
\NormalTok{        x_logical[n] =}\StringTok{ "even"}
\NormalTok{        n=n}\OperatorTok{+}\DecValTok{1}
\NormalTok{    \}}
    \CommentTok{#odd}
    \ControlFlowTok{else}
\NormalTok{    \{}
\NormalTok{      x_logical[n] =}\StringTok{ "odd"}
\NormalTok{      n=n}\OperatorTok{+}\DecValTok{1}
\NormalTok{    \}}
        
\NormalTok{\}}

\NormalTok{x_logical}
\end{Highlighting}
\end{Shaded}

\begin{verbatim}
##       [,1]   [,2]   [,3]   [,4]   [,5]   [,6]   [,7]   [,8]   [,9]   [,10] 
##  [1,] "odd"  "odd"  "odd"  "odd"  "odd"  "odd"  "odd"  "odd"  "odd"  "odd" 
##  [2,] "even" "even" "even" "even" "even" "even" "even" "even" "even" "even"
##  [3,] "odd"  "odd"  "odd"  "odd"  "odd"  "odd"  "odd"  "odd"  "odd"  "odd" 
##  [4,] "even" "even" "even" "even" "even" "even" "even" "even" "even" "even"
##  [5,] "odd"  "odd"  "odd"  "odd"  "odd"  "odd"  "odd"  "odd"  "odd"  "odd" 
##  [6,] "even" "even" "even" "even" "even" "even" "even" "even" "even" "even"
##  [7,] "odd"  "odd"  "odd"  "odd"  "odd"  "odd"  "odd"  "odd"  "odd"  "odd" 
##  [8,] "even" "even" "even" "even" "even" "even" "even" "even" "even" "even"
##  [9,] "odd"  "odd"  "odd"  "odd"  "odd"  "odd"  "odd"  "odd"  "odd"  "odd" 
## [10,] "even" "even" "even" "even" "even" "even" "even" "even" "even" "even"
\end{verbatim}

\hypertarget{part-3}{%
\subsection{Part 3}\label{part-3}}

\hypertarget{q5}{%
\subsubsection{Q5}\label{q5}}

Write a for loop that iterates over the column names of the
\texttt{mtcars} data set and prints the name with the number of
characters in the column name in parentheses.

Example output: \texttt{"mpg\ (3)"}.

Use the following functions, and any others you wish: \texttt{paste} and
\texttt{nchar} functions.

Note: \texttt{nchar} function counts how many characters are in a
string.

\begin{Shaded}
\begin{Highlighting}[]
\KeywordTok{data}\NormalTok{(mtcars)}

\CommentTok{#initialize the two variables}
\NormalTok{col_names <-}\StringTok{ }\KeywordTok{colnames}\NormalTok{(mtcars) }\CommentTok{# listn of all col names}
\NormalTok{names_count <-}\StringTok{ }\KeywordTok{nchar}\NormalTok{(col_names) }\CommentTok{#counts chats in col name}

\CommentTok{#goes down col_names, pastes name then character count}
\ControlFlowTok{for}\NormalTok{ (i }\ControlFlowTok{in} \KeywordTok{seq_along}\NormalTok{(col_names))\{}
    \KeywordTok{print}\NormalTok{(}\KeywordTok{paste}\NormalTok{(col_names[i], }\StringTok{" ("}\NormalTok{,(names_count[i]), }\StringTok{")"}\NormalTok{,}\DataTypeTok{sep =} \StringTok{""}\NormalTok{))}

\NormalTok{\}}
\end{Highlighting}
\end{Shaded}

\begin{verbatim}
## [1] "mpg (3)"
## [1] "cyl (3)"
## [1] "disp (4)"
## [1] "hp (2)"
## [1] "drat (4)"
## [1] "wt (2)"
## [1] "qsec (4)"
## [1] "vs (2)"
## [1] "am (2)"
## [1] "gear (4)"
## [1] "carb (4)"
\end{verbatim}

\hypertarget{q6}{%
\subsubsection{Q6}\label{q6}}

Use a while loop to investigate the number of terms required until the
product of

1⋅2⋅3⋅4⋅\ldots{}

reaches above 1 million. How many elements is needed to reach 1 million.

\begin{Shaded}
\begin{Highlighting}[]
\CommentTok{#initialize beginning values, gets factorial of 1}
\NormalTok{product <-}\StringTok{ }\DecValTok{0}
\NormalTok{a <-}\StringTok{ }\DecValTok{1}
\NormalTok{b <-}\StringTok{ }\DecValTok{1}
\ControlFlowTok{for}\NormalTok{ (i }\ControlFlowTok{in}\NormalTok{ b)}
\ControlFlowTok{while}\NormalTok{ (product }\OperatorTok{<=}\StringTok{ }\DecValTok{1000000}\NormalTok{) }\CommentTok{#keeps going until it hits the first product above 1,000,000}
\NormalTok{  \{}
\NormalTok{      product[i] <-}\StringTok{ }\KeywordTok{factorial}\NormalTok{(a}\OperatorTok{+}\NormalTok{b}\DecValTok{-1}\NormalTok{)}\OperatorTok{/}\KeywordTok{factorial}\NormalTok{(a}\DecValTok{-1}\NormalTok{)}
      
      \CommentTok{#adds element if product hasn't reached 1,000,000}
      \ControlFlowTok{if}\NormalTok{ (product }\OperatorTok{<=}\StringTok{ }\DecValTok{1000000}\NormalTok{) }
\NormalTok{      b <-}\StringTok{ }\NormalTok{b }\OperatorTok{+}\StringTok{ }\DecValTok{1}
\NormalTok{  \}}
\NormalTok{product}
\end{Highlighting}
\end{Shaded}

\begin{verbatim}
## [1] 3628800
\end{verbatim}

\begin{Shaded}
\begin{Highlighting}[]
\NormalTok{b}
\end{Highlighting}
\end{Shaded}

\begin{verbatim}
## [1] 10
\end{verbatim}

10 elements are needed to reach the first value greater than or equal to
1 million. The first product after breaking 1,000,000 is 3,628,800

\hypertarget{q7}{%
\subsubsection{Q7}\label{q7}}

Write a while loop that prints out standard random normal numbers (use
\texttt{rnorm}) but stops (breaks) if you get a number bigger than 1.

\begin{Shaded}
\begin{Highlighting}[]
\KeywordTok{set.seed}\NormalTok{(}\DecValTok{245424}\NormalTok{)}

\NormalTok{r <-}\StringTok{ }\DecValTok{0}
\ControlFlowTok{while}\NormalTok{(r}\OperatorTok{<=}\DecValTok{1}\NormalTok{) }\CommentTok{#stops at first value above 1}
\NormalTok{\{}
\NormalTok{  r <-}\StringTok{ }\KeywordTok{rnorm}\NormalTok{ (}\DataTypeTok{n =} \DecValTok{1}\NormalTok{)}
  \KeywordTok{print}\NormalTok{(r) }
\NormalTok{\}}
\end{Highlighting}
\end{Shaded}

\begin{verbatim}
## [1] 0.5610392
## [1] -1.913313
## [1] -2.041071
## [1] 1.838673
\end{verbatim}

\hypertarget{q8}{%
\subsubsection{Q8}\label{q8}}

Using a while loop, generate a 100 positive values from a standard
normal distribution. Your vector should look similar to the vector
below.

\begin{verbatim}
##   [1] 1.657517033 0.187654732 0.420571550 2.400824012 1.778782240 0.139411590
##   [7] 2.826552130 0.194471297 0.947344730 0.977882326 1.814398672 0.003000291
##  [13] 0.728592024 0.949425915 1.101056596 0.041289591 1.009828176 0.550637293
##  [19] 0.014740967 0.111447329 0.572745045 0.800523955 0.734825521 0.865536592
##  [25] 0.490190027 0.623598804 2.080232886 0.204695714 1.997787659 0.575125862
##  [31] 0.333256358 0.497321518 0.736505236 0.815510063 1.095037944 0.503035171
##  [37] 1.037377878 0.111853558 0.166957790 0.465937984 0.483296281 1.020587002
##  [43] 0.417957347 0.546884173 0.512819917 0.205997706 0.640649401 1.558822629
##  [49] 0.140238681 1.406808863 1.592408776 1.567603000 0.524083433 0.179271968
##  [55] 0.185835934 0.347468756 0.736258400 1.024982780 0.183575592 1.325723900
##  [61] 0.235419372 1.020786063 1.100859227 0.437973656 1.175132331 1.781096814
##  [67] 0.623395832 0.019826157 0.843880907 1.011377922 0.980958225 0.067721184
##  [73] 0.260010939 0.857796852 1.529206187 1.078207047 1.646897850 0.143122481
##  [79] 0.301934782 0.232940022 0.646350604 1.906270548 2.127012382 0.336091477
##  [85] 0.583577703 0.432255653 0.438776811 0.927996234 0.521831032 1.896265090
##  [91] 1.130087197 1.204216202 0.806713542 0.773139443 0.653696722 0.325332444
##  [97] 0.744729166 0.133474009 0.245825896 0.709354972
\end{verbatim}

\begin{Shaded}
\begin{Highlighting}[]
\NormalTok{pos_}\DecValTok{100}\NormalTok{ <-}\StringTok{ }\KeywordTok{c}\NormalTok{() }\CommentTok{#initialize empty vector}
\NormalTok{i <-}\StringTok{ }\DecValTok{1}
\ControlFlowTok{while}\NormalTok{ (i }\OperatorTok{<=}\StringTok{ }\DecValTok{100}\NormalTok{) }\CommentTok{#limit to 100 }
\NormalTok{\{}
\NormalTok{  p =}\StringTok{ }\KeywordTok{rnorm}\NormalTok{(}\DecValTok{1}\NormalTok{,}\DecValTok{0}\NormalTok{,}\DecValTok{1}\NormalTok{)}
  \ControlFlowTok{if}\NormalTok{ (p}\OperatorTok{>}\DecValTok{0}\NormalTok{) }\CommentTok{#only keep positive values}
\NormalTok{  \{}
\NormalTok{    pos_}\DecValTok{100}\NormalTok{[i] <-}\StringTok{ }\NormalTok{p}
\NormalTok{    i =}\StringTok{ }\NormalTok{i }\OperatorTok{+}\StringTok{ }\DecValTok{1}
\NormalTok{  \}}
\NormalTok{\}}
\NormalTok{pos_}\DecValTok{100}
\end{Highlighting}
\end{Shaded}

\begin{verbatim}
##   [1] 1.188034197 0.949146825 0.129378372 0.553509081 0.099015507 0.853672570
##   [7] 0.593074006 1.520092833 0.284597952 0.115915576 0.666915426 0.643007739
##  [13] 0.927475409 0.307312538 0.255772081 0.415701983 0.159390943 1.324830716
##  [19] 0.706174833 0.004279386 0.355707127 1.274162791 0.888278360 0.701241808
##  [25] 0.496836864 0.324962472 1.065868928 1.000610496 0.151053538 0.773000780
##  [31] 1.934408840 0.446479191 1.459472006 1.733671891 0.751582218 0.144590900
##  [37] 0.395555559 0.091011968 0.764085045 0.271785926 0.950612761 0.735703341
##  [43] 0.586689384 0.872917942 0.303641234 1.021710236 1.727758369 0.409899642
##  [49] 0.307786673 1.555953707 1.858755311 0.712110896 0.810071270 0.146760115
##  [55] 1.191032546 2.541655278 0.359472436 0.617059959 0.010177544 0.429283415
##  [61] 1.339972469 0.675018365 0.410424408 1.444126083 0.903884746 1.089255851
##  [67] 2.020245336 0.390945638 1.192977184 0.913012599 0.421856824 0.599863919
##  [73] 0.933449834 0.064722510 0.084224556 1.405517318 0.150032758 0.747948255
##  [79] 1.252908259 1.284391595 0.422438686 1.519765150 1.249504088 0.796970332
##  [85] 1.890234120 0.914327873 1.185136623 0.591065948 0.789479860 0.507887606
##  [91] 0.115853587 2.401901898 0.642941379 1.314198182 1.295976725 1.362782129
##  [97] 0.660369723 0.836101910 1.650260757 0.671361501
\end{verbatim}

\hypertarget{part-4}{%
\subsection{Part 4}\label{part-4}}

\hypertarget{q9}{%
\subsubsection{Q9}\label{q9}}

Watch the following video
\url{https://www.youtube.com/watch?v=Mv9NEXX1VHc}:

What is recursion?

Recursion is when data can be passed through a function and passed
through itself until conditions for termination (base condition) are
met, in which all data up until putting it out will be returned.

\hypertarget{q10}{%
\subsubsection{Q10}\label{q10}}

Watch the following video
\url{https://www.youtube.com/watch?v=HXNhEYqFo0o}:

What is the difference between loops and recursion?

A loop will continue with the action until it reaches the condition for
``ending the list'', meanwhile recursion will return data to the
function if the base condition to terminate isn't met.

\end{document}
